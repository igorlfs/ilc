\documentclass[leqno]{article} %% Alinhe num equações à esquerda
%%\usepackage{etoolbox}
%%\usepackage{amssymb}
\usepackage{hyperref}       %% Use links
\usepackage{indentfirst}    %% Indente o primeiro parágrafo
\usepackage{amsfonts}       %% Conjuntos
\usepackage{amssymb}        %% QED
\usepackage{amsmath}
\usepackage{enumitem}
\let\biconditional\leftrightarrow
\setlist{  listparindent=\parindent }
\AtBeginEnvironment{quote}{\par\singlespacing\small} %% Faz citações terem formatação diferente
\author{Igor Lacerda Faria}
\begin{document}

\title{\textbf{Lista de Exercícios 0.4}}
\date{%
    \( ^1 \)Departamento de Ciência da Computação - Universidade Federal de Minas Gerais (UFMG) - Belo Horizonte - MG - Brasil \\ [3ex]
    \href{mailto:igorlfs@ufmg.br}{\nolinkurl{igorlfs@ufmg.br}}
}
\maketitle
\begin{enumerate}

    \item

        \begin{enumerate}

            \item Um argumento lógico é um tipo de argumento composto por um conjunto de \( n \) proposições lógicas e que possui a propriedade de que se as \( n - 1 \) primeiras proposições (as chamadas premissas) forem verdadeiras, então a proposição restante (chamada conclusão) também é verdadeira. O exemplo canônico disso é a seguinte colocação:

                \begin{itemize}

                    \item[p:] Todo homem é mortal.

                    \item[q:] Sócrates é homem.

                \end{itemize}

                Partindo do pressuposto que \( p \) e \( q \) são verdadeiras, podemos concluir \( r: \) Sócrates é mortal.

            \item Um argumento válido é aquele em que ao se tomar suas premissas como verdadeiras, sua conclusão é \textbf{necessariamente} verdaeira. Por outro lado, um argumento inválido é aquele que \textit{apesar de poder ser verdadeiro,} sua veracidade \textbf{não é garantida} pela veracidade de suas premissas.

            \item É possível que argumento válido tenha uma conclusão falsa ao se tomar como verdadeira uma premissa que na realidade é falsa. A validade do argumento não diz respeito ao conteúdo das proposições em si, mas à sua estrutura dentro da lógica.

            \item Uma falácia lógica é uma colocação que \textit{parece} um argumento válido mas não é. Devido à minha falta de criatividade hoje, irei apenas citar os exemplos vistos em aula: 

                \begin{itemize}

                    \item \textbf{Falácia da afirmação da conclusão} 

                        \noindent Exemplo: Todo natural é inteiro. -1 é inteiro. Logo -1 é natural.

                        Aqui temos algo como: \( \forall x : (P(x) \to Q(x)) \), \( Q(a) \) e concluímos \( P(a) \). Sabendo que \( Q(a) \), no caso, que dado número é inteiro, não podemos afirmar nada do antecedente: ele pode tanto ser verdadeiro como falso. No exemplo, o número pode ser natural ou negativo.

                    \item \textbf{Falácia da negação do antecedente} 

                        \noindent Exemplo: Todo natural é inteiro. -1 não é natural. Logo -1 não é inteiro.

                        Aqui temos algo como: \( \forall x : (P(x) \to Q(x)) \), \( \neg P(a) \) e concluímos \( \neg Q(a) \). Negando o antencedente, o condicional não introduz nenhuma informação nova, então não podemos afirmar nada sobre \( Q(a) \). 

                \end{itemize}

        \end{enumerate}

    \item 
        \begin{enumerate}
            \item Adição conjuntiva: \( p \Rightarrow p \lor q \)
            \item Simplificação conjuntiva: \( p \land q \Rightarrow  p \)
            \item Modus Ponens: \( p \to q, p \Rightarrow q \)
            \item Modus Tollens: \( p \to q, \neg q \Rightarrow \neg p \)
            \item Silogismo hipotético: \( p \to q, q \to r \Rightarrow p \to r\)
        \end{enumerate}

    \item

        \begin{enumerate}

            \item 

                \begin{itemize}

                    \item[p:] Eu tiro o dia de folga

                    \item[q:] Chove no dia

                    \item[r:] Neva no dia

                \end{itemize}

                Para \( T \) terça e \( Q \) quinta, temos:

                \begin{align}
                     &\forall x : (p(x) \to (q(x) \lor r(x))) &\textrm{Premissa} \\
                     &p(T) \lor p(Q) &\textrm{Premissa} \\
                     &\neg q(T) \land \neg r(T) \footnote{Se fez sol, não choveu nem nevou} &\textrm{Premissa} \\
                     &\neg r(Q) &\textrm{Premissa} \\
                     &\neg p(T) &\textrm{(1), (3), Modus Tollens} \\
                     &p(Q) &\textrm{(2), (5), Silogismo Disjuntivo} \\
                     &q(Q) \lor r(Q) &\textrm{(6), (1), Modus Ponens Universal} \\
                     &q(Q) &\textrm{(7), (4), Silogismo Disjuntivo}
                \end{align}

                \( \therefore \) A pessoa em questão tirou folga somente na quinta, e choveu nesse dia.

            \item 

                \begin{itemize}

                    \item[C:] Eu como comida apimentada

                    \item[S:] Eu tenho sonhos estranhos

                    \item[T:] Troveja enquanto durmo

                \end{itemize}

                \setcounter{equation}{0}% Restart equation counter
                \begin{align}
                     &C \to S &\textrm{Premissa} \\
                     &T \to S&\textrm{Premissa} \\
                     &\neg S &\textrm{Premissa} \\
                     &\neg C &\textrm{Modus Tollens: (1), (3)} \\
                     &\neg T &\textrm{Modus Tollens: (2), (3)}
                \end{align}

                \( \therefore \) Não trovejou nem choveu nessa noite e nem comi comida apimentada.

            \item

                \begin{itemize}

                    \item[E:] Eu sou esperto

                    \item[S:] Eu sou sortudo

                    \item[L:] Eu ganho na loteria

                \end{itemize}

                \setcounter{equation}{0}% Restart equation counter
                \begin{align}
                     &E \lor S &\textrm{Premissa} \\
                     &\neg S &\textrm{Premissa} \\
                     &S \to L&\textrm{Premissa} \\
                     &E &\textrm{Silogismo disjuntivo (1), (2)}
                \end{align}

                \( \therefore \) Eu sou esperto.
                %% Perceba que não posso afirmar que não sou sortudo, apesar de parecer muito o caso.

            \item

                \begin{itemize}

                    \item[r(x):] x é roedor

                    \item[c(x):] x rói a própria comida

                    \item[T:] \( \forall x: (r(x) \to c(x)) \)

                \end{itemize}

                Para \( R \) ratos, \( C \) coelhos e \( M \) morcegos, temos:

                \setcounter{equation}{0}% Restart equation counter
                \begin{align}
                     &\forall x: (r(x) \to c(x)) &\textrm{Premissa} \\
                     &r(R) &\textrm{Premissa} \\
                     &\neg c(C)&\textrm{Premissa} \\
                     &\neg r(M)&\textrm{Premissa} \\
                     &c(R) &\textrm{Modus Ponens Universal: (1), (2)} \\
                     &\neg r(C) &\textrm{Modus Tollens Universal: (1), (3)}
                \end{align}

                \( \therefore \) Ratos roem sua própria comida e coelhos não são roedores.
                %% Perceba que não posso afirmar que morcegos não roem a própria comida (tenho a negação do antecedente).
        \end{enumerate}

    \item 

        \begin{enumerate}

            \item Modus Ponens Universal;
            %% Uau, o livro tem toda uma argumentação pra isso

            \item Generalização existencial, instanciação universal.
            %% Uau, o livro tem toda uma argumentação pra isso

        \end{enumerate}

    \item 

        \begin{enumerate}

            \item Esse argumento está correto. Se \( P(x) \) para todo \( x \) no domínio então \( P(c) \) para um \( c \) particular.

            \item Incorreto. Natasha pode estar cursando Matemática Discreta mas ser de outro curso, sem contradizer a condição de que todo estudante de CC faz a matéria.

            \item Incorreto. Raciocínio análogo ao anterior. Se não vale a condição, o condicional não dá nenhuma informação.

            \item Correto. Modus Tollens.

        \end{enumerate}

    \item 

        \begin{enumerate}

            \item Inválido. Isso é um exmeplo de falácia de \textit{afirmação da conclusão}. Saber a conclusão não dá informações sobre o antecedente.

            \item Válido. Modus Tollens.

            \item Inválido. Isso é um exemplo de falácia de \textit{negaçõa do antecedente}. Negar o antecedente torna a condicional ``inútil''.

        \end{enumerate}

    \item O erro nesse argumento está no passo 6, de adição conjuntiva de (3) e (5). O problema aqui, que muitas vezes é cometido por falta de atenção, é que o \( c \) de \( P(c) \) não é, necessariamente, o mesmo \( c \) de \( Q(c) \). Para evitar esse tipo de erro, é útil usar uma notação diferente para cada variável dentro de um \textit{escopo} (como uma questão ou item de questão). Por exemplo, pode-se usar índices: \( \exists c_1 : P(c_1) \) e \( \exists c_2 : Q(c_2) \). Se (6) estivesse correto, então a dedução de (7) seria válida.

\end{enumerate}

\end{document}
