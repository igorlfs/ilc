\documentclass{article} 
\usepackage{hyperref}       %% Use links
\usepackage{indentfirst}    %% Indente o primeiro parágrafo
\usepackage{amsfonts}       %% Conjuntos
\usepackage{amssymb}        %% QED
\usepackage{multicol}       %% Mais de uma coluna
\usepackage{amsmath}
\usepackage{enumitem}
\usepackage[T1]{fontenc}        % Encoding para português 
\usepackage{lmodern}            % Conserta a fonte para PT
\usepackage[portuguese]{babel}  % Português
\usepackage{hyphenat}           % Use hifens corretamente

\hyphenation{mate-mática recu-perar}

\newcommand{\qed}{\hfill\rule{1ex}{1ex}}

\author{Igor Lacerda Faria}

\begin{document}

\title{\textbf{Lista de Exercícios 0.7}}

\date{%
    \( ^1 \)Departamento de Ciência da Computação - Universidade Federal de Minas Gerais (UFMG) - Belo Horizonte - MG - Brasil \\ [3ex]
    \href{mailto:igorlfs@ufmg.br}{\nolinkurl{igorlfs@ufmg.br}}
}

\maketitle

\section*{Revisão}

\begin{enumerate}

    \item 

        \begin{enumerate}

            \item Um conjunto é enumerável se possui uma bijeção com o conjunto dos números naturais ou algum subconjunto deste. Um conjunto é não enumerável se não é enumerável.

            \item O Teorema de Schröder-Bernstein afirma que para estabelecer que dois conjuntos \( A, B \) possuem a mesma cardinalidade é suficiente mostrar que existe uma injeção de \( A \) para \( B \) e uma injeção de \( B \) para \( A \). Isso é bem legal, porque encontrar uma sobrejeção é um processo particularmente mais trabalhoso.

            \item Suponha por contradição que listemos todos os programas de computador. Ou seja, estamos listando \textit{strings} de tamanho arbitrário (aquelas que geram programas em alguma linguagem). Podemos criar um mapeamento para um subconjunto específico de \( \mathbb{N} \), que será usado como \textit{base} na nossa diagonalização de Cantor. Listamos os itens e seus mapeamentos para os naturais, criando uma estrutura parecida com uma matriz. Agora é possível criar um programa que difere para pelo menos um natural nas saídas.

        \end{enumerate}

        \section*{Exercícios}

    \item 

        \begin{enumerate}

            \item \( f(x) \) não está definida para \( x=0 \).

            \item \( f(x) \) não está definida para \( x=-1 \).

            \item \( f(x) \) não é função pois mapeia duas saídas para uma única entrada.

        \end{enumerate}

    \item \textit{Mas olha, eu poderia definir um domínio arbitrário...}

        \begin{enumerate}

            \item \textbf{Domínio:} \( \mathbb{Z} \); \textbf{Imagem:} \( {0,1,2,3,4,5,6,7,8,9} \).

            \item \textbf{Domínio:} \( \mathbb{Z} \); \textbf{Imagem:} \( \mathbb{N}^* \).

            \item \textbf{Domínio:} conjunto de strings binárias; \textbf{Imagem:} \( \mathbb{N} \).

        \end{enumerate}

    \item \textit{Ordenação levemente deslocada}

        \begin{multicols}{2}
            \begin{enumerate}
                \item 0
                \item -1
                \item 3
                \item 1
            \end{enumerate}
        \end{multicols}

    \item \textit{De \( \mathbb{Z} \) para \( \mathbb{Z} \)}

        \begin{enumerate}

            \item bijetiva

            \item não injetiva e não sobrejetiva

            \item bijetiva

            \item sobrejetiva (possível atingir todo inteiro), mas não injetiva (1 e 2 mapeiam para 1)

        \end{enumerate}

    \item 

        \begin{enumerate}

            \item 1

            \item \( 0 < x < 1 \)

            \item \( x > 2 \)

        \end{enumerate}

    \item Se \( x=0 \), \( \lceil 0 \rceil = \lceil -0 \rceil = \lfloor 0 \rfloor = \lfloor -0 \rfloor = 0 \). Se \( x \neq  0 \), ou \( x \) é inteiro ou \( \exists n, \epsilon \mid n \in \mathbb{Z}, 0 < \epsilon < 1 \mid x = n + \epsilon \). Se \( x \) é inteiro, \( \lfloor -x \rfloor = -x = - \lceil x \rceil\). Caso contrário, \( \lfloor -x \rfloor = \lfloor -(n + \epsilon)  \rfloor = \lfloor - n - \epsilon  \rfloor = -(n +1)\) e \( - \lceil x \rceil = - \lceil n + \epsilon \rceil - (n+1)\). A outra proposição segue de forma análoga.

    \item 

        \begin{enumerate}

            \item 1, 4, 27, 256
            \item 0, 1, 2, 3

        \end{enumerate}

    \item 

        \begin{enumerate}

            \item -1, -2, -2, 8, 88, 656, 4'912, 40'064 362'368, 3'627'776

            \item 3, 6, 12, 24, 48, 96, 192, 384, 768, 1'536

            \item 2, 4, 6, 10, 16, 26, 42, 68, 110, 178

            \item 2, 4, 4, 6, 5, 4, 4, 4, 4, 3

        \end{enumerate}

    \item 

        \begin{enumerate}

            \item \( f(1) = 3, f(n) = f(n-1) + 2n-1, \forall n > 1 \), 

                \textbf{Próximos termos:} 123, 146, 171.

            \item \( f(1) = 7, f(n) = f(n-1) + 4, \forall n > 1 \)

                \textbf{Próximos termos:} 47, 51, 55.

            \item \( f(n) = (n)_2 \) (base 2)

                \textbf{Próximos termos:} 1100, 1101, 1110

            \item O primeiro termo é 1, que aparece só uma vez. Depois, escolha o próximo número primo e repita-o \( 2n + 1 \) vezes, em que \( n \) corresponde a posição do i-ésimo primo.

                \textbf{Próximos termos:} 7, 7, 7

            \item \( f(1) = 0, f(n) = 3 \cdot f(n-1) + 2 \)

                \textbf{Próximos termos:} 59'048, 177'146, 531'440

        \end{enumerate}

    \item 

        \begin{enumerate}

            \item 0

            \item Calculando o resto de 99 por 4, que é 3, temos que o resultado é 1. Perceba que para \( i = 0 \), o resultado é 1, para \( i = -1 \) é -1. No geral, vamos precisar de dois ímpares aparecendo para mudar o sinal, então basta olhar o resto por 4.

            \item igual a \( 2^{12} \), que é 4096. Atenção à borda: \( i = 1 \Rightarrow 2^{1+1} = 4 \).

        \end{enumerate}

    \item 

        \begin{enumerate}

            \item Infinito Enumerável. \( \mathbb{N} \to \mathbb{Z}^- : f(n) = -n\)

            \item Infinito Enumerável. \( 1 \to 2, 2 \to -2, 3 \to 4, 4 \to -4, \mathellipsis \)

            \item Não Enumerável.

            \item Infinito Enumerável. \( 1 \to 7, 2 \to -7, 3 \to 14, 4 \to -14, \mathellipsis \)

        \end{enumerate}

    \item Isso é como mapear naturais nos inteiros. A nova posição da pessoa no quarto \( n \) do hotel inicial é \( f(n) = \lfloor \frac{n \cdot (-1)^n}{2} \rfloor \). Por exemplo, se a pessoa está no quarto 1, seu novo quarto é o quarto 0 (um quarto adicional do segundo hotel). Se a pessoas está no quarto 2, seu novo quarto é 1. Em 3, -1 (o quarto seguinte do outro hotel), e assim sucessivamente.

    \item 

        \begin{enumerate}

            \item \( A = [0, 1/2], B = [1/2,1] \Rightarrow A \cap B = \{ 1/2 \}\) 

            \item \( A = [0,1] \cup \mathbb{N}, B = [1,2] \cup \mathbb{N} \Rightarrow A \cap B = \mathbb{N} \)

            \item \( A = B = \mathbb{R} \)

        \end{enumerate}

    \item Isso segue o mesmo princípio que a demonstração de que \( \mathbb{Q} \) é enumerável, mas usa conjuntos menores. Listamos os naturais\footnote{Inteiros Positivos, se quiser} em linhas e colunas, e criamos os pares. Em seguida, caminhamos nas diagonais (é algo bem chatinho de desenhar pelo \LaTeX), fazendo a associação com \( \mathbb{N} \).

    \item Consideramos a seguinte função \( f(x) \), que associa a cada número no intervalo (0,1)\footnote{Remover o 0 de [0,1) mantém a inumerabilidade} um número no intervalo (a,b):

                \[ f(x) = (1-x)a + xb \]

                Essa função é injetiva. Suponha por contradição que não seja. Então \( \exists x_1, x_2 \mid x_1 \neq x_2 \land f(x_1) = f(x_2) \). Mas assim: 

                \[ a - x_1 \cdot a + x_1 \cdot b = a - x_2 \cdot a + x_2 \cdot b \Rightarrow x_1 = x_2 \]

                Então já temos uma injeção. Similarmente temos:

                \[ g(x) = (b-a)x + a \]

                Que pode-se mostrar que é injetiva (analogamente à função anterior), mapeando (0,1) em (a,b). E pelo Teorema de Schröder-Bernstein, como existe uma injeção nos dois sentidos, ambos tem a mesma cardinalidade, o que significa que (a,b) é não enumerável.

\end{enumerate}

\end{document}
