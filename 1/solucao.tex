\documentclass[12pt]{article}
\usepackage{indentfirst} %% Indente o primeiro parágrafo
\usepackage{amsfonts} %% Conjuntos
\usepackage{setspace} %% Necessário para usar as quotes
%%\usepackage{etoolbox} ?
%%\usepackage{amssymb}
\usepackage{amssymb} %% QED
\usepackage{enumitem}
\let\biconditional\leftrightarrow
\setlist{  listparindent=\parindent }
\AtBeginEnvironment{quote}{\par\singlespacing\small} %% Faz citações terem formatação diferente
\title{Lista 0.1}
\author{Igor Lacerda}
\begin{document}
\maketitle
\section*{Nota}
Fazendo mais uma vez em \LaTeX. Em algum momento começo a usar papel.

Algumas notações que decidi adotar na lista: \textit{representar verdadeiro com 1 e falso com 0.}

\section*{Revisão (1) + Exercícios}
\begin{enumerate}
    \item 
        \begin{enumerate}
            \item Uma proposição é uma frase declarativa (afirmação) que pode assumir somente um entre os dois valores, normalmente chamados de verdadeiro e falso.

            \item Uma proposição condicional é um tipo de proposição composta (ou seja, uma proposição contruída a partir do uso de conectivos), formada a partir de duas proposições \( p,q \) e o conectivo \( \to \):

                \[ p \to q \]

                Ela é definida como falsa \textit{somente se} \( p \) for verdadeira e \( q \) for falsa. Desse modo, pode ser entendida como uma abreviação da proposição composta 

                \[ \neg p \lor q \] 

                Analisemos esse fato com o uso de uma \textit{tabela verdade}:
                \begin{center}
                    \begin{tabular}{||c c c c c||} 
                        \hline
                        \( p \)  & \( q \)  & \( p \to q \)  & \( \neg p \) & \( \neg p \lor q\)  \\ [0.5ex] 
                        \hline\hline
                        0 & 0 & 1 & 1 & 1 \\ 
                        \hline
                        0 & 1 & 1 & 1 & 1 \\
                        \hline
                        1 & 0 & 0 & 0 & 0 \\
                        \hline
                        1 & 1 & 1 & 0 & 1 \\ [1ex]
                        \hline
                    \end{tabular}
                \end{center}

                Vemos que a terceira e a última coluna possuem os mesmo valores, logo as proposições são equivalentes. Para entender melhor a relação criada na condicional, podemos entender que, como visto nas aulas, ela é false se \textit{q é ``menos verdadeira'' que p}.

            \item 
                \begin{itemize}
                    \item q se p;
                    \item se p então q;
                    \item p implica q;
                    \item p é suficiente para q;
                    \item q é necessário para p.
                \end{itemize}
        \end{enumerate}
    \item 
        \begin{enumerate}
            \item É proposição (verdaeira);
            \item É proposição (falsa);
            \item É proposição (verdadeira);
            \item É proposição (falsa);
            \item \textbf{Não} é proposição;
            \item \textbf{Não} é proposição;
        \end{enumerate}
    \item Escreva:

        \begin{center}
            p: “Nadar na praia em Nova Jersey é permitido”

            q: “Foram descobertos tubarões perto da praia”, 
        \end{center}

        \begin{enumerate}
            \item \( \neg q \): \textbf{Não} foram descobertos tubarões perto da praia.

            \item \( p \land q \): Nadar na praia em Nova Jersey é permitido \textbf{e} foram descobertos tubarões perto da praia.

            \item \( \neg p \lor q \): Nadar na praia em Nova Jersey \textbf{não} é permitido \textbf{ou} foram descobertos tubarões perto da praia

            \item \( p \to \neg q \): Se nadar na praia em Nova Jersey é permitido, então não foram descobertos tubarões perto da praia

            \item \( p \biconditional q \): Nadar na praia em Nova Jersey é permitido se, e somente se não foram descobertos tubarões perto da praia.

            \item \( \neg p \land (p \lor \neg q) \) Nadar na praia em Nova Jersey \textbf{não} é permitido \textbf{e} nadar na praia em Nova Jersey é permitido \textbf{ou} \textbf{não} foram encontrados descobertos tubarões perto da praia
        \end{enumerate}

    \item Escreva:
        \begin{center}
            p: “Está abaixo de zero”

            q: “Está nevando”
        \end{center}
        \begin{enumerate}
            \item \( p \land q \)
            \item \( p \land \neg q \) 
            \item \( \neg p \land \neg q \) 
            \item \( q \lor p \) 
            \item \( p \to q \) 
            \item \( p \oplus q \) 
            \item \( p \biconditional q \) 
        \end{enumerate}

    \item Dados:
        \begin{center}
            p: “Ursos-cinzentos são vistos na área.”

            q: “Fazer caminhada na trilha é seguro.”

            r: “As bagas estão maduras ao longo da trilha.”
        \end{center}
        \begin{enumerate}
            \item \( r \land \neg p \) 
            \item \( (\neg p \land q) \land r\)
            \item \( (r \to q) \biconditional (\neg p) \) 
            \item \( (\neg q) \land (\neg p \land r) \) 
            \item \( (\neg r \land \neg p) \to q \) 
            \item \( (r \land p) \to \neg q  \) 
        \end{enumerate}
    \item 
        \begin{enumerate}
            \item Falsa. Hipótese verdadeira \( 1 + 1 = 2 \), mas tese falsa \( 2 + 2 = 5\);
            \item Verdadeira. Hipótese falsa \( 1 + 1 = 3 \);
            \item Verdadeira. Hipótese falsa \( 1 + 1 = 3 \);
            \item Verdaderia. Hipótese falsa (macacos podem voar).
        \end{enumerate}
    \item 
        \begin{enumerate}
            \item Para cursar matemática discreta, você deve ter tido cálculo ou um curso de ciência da computação.

                \textbf{INCLUSIVO} Para cursar matemática discreta, você deve ter tido cálculo ou um curso de ciência da computação ou ambos.

                \textbf{EXCLUSIVO} Para cursar matemática discreta, você deve ter tido cálculo ou um curso de ciência da computação mas não pode ter feito ambos.

                Eu acredito que o autor quis usar o ou \textbf{inclusivo}. 

            \item Quando você compra um novo carro da Companhia Acme Motor, você pega de volta \$ 2.000 ou um empréstimo de 2\%.

                \textbf{INCLUSIVO} Quando você compra um novo carro da Companhia Acme Motor, você pega de volta \$ 1.000 ou um empréstimo de 2\% ou pega ambos os benefícios.

                \textbf{EXCLUSIVO}  Quando você compra um novo carro da Companhia Acme Motor, você pega de volta \$ 2.000 ou um empréstimo de 2\% mas não pode pegar ambos os benefícios.

                Eu acredito que o autor quis usar o ou \textbf{exclusivo}. 
        \end{enumerate}
    \item \textit{Nota do estudante:} além do ``se'' \textit{tradicional} do se ... então, foi usado ``se'' para \textit{se flexionar os verbos.}
        \begin{enumerate}
            \item Se o vento sopra do nordeste, então neva.
            \item Se se chegou ao topo de Long's Park, então foram andadas 8 milhas.
            \item Se se é famoso mundialmente, então se é efetivado como professor. 
            \item Se você comprou um seu aparelho de som há menos de 90 dias, então a sua garantia é válida.
            \item Se a água não estiver muito fria, então Jan nadará.
        \end{enumerate}
    \item 
        %%\textbf{TODO:} incluir isso no resumo;
        \begin{enumerate}
            \item Se nevar hoje, esquiarei amanhã.

                \textbf{OPOSTA} Se esquiarei amanhã, então neva hoje.

                \textbf{CONTRAPOSITIVA} Se não esquiarei amanhã então não neva hoje.

                \textbf{INVERSA} Se não nevar hoje, então não esquiarei amanhã.

            \item Eu venho à aula sempre que há uma prova.

                \textbf{OPOSTA} Se eu venho à aula então há uma prova.

                \textbf{CONTRAPOSITIVA} Se não venho à aula então não há uma prova.

                \textbf{INVERSA} Se não há uma prova então eu não venho à aula.
        \end{enumerate}
    \item 
        \begin{enumerate}
            \item \( (p \lor \neg q) \to q  \) 
                \begin{center}
                    \begin{tabular}{||c c c c c||} 
                        \hline
                        \( p \)  & \( q \)  & \( \neg q\)  & \( p \lor \neg q \) & \( (p \lor \neg q) \to q  \)  \\ [0.5ex] 
                        \hline\hline
                        0 & 0 & 1 & 1 & 0 \\ 
                        \hline
                        0 & 1 & 0 & 0 & 1 \\
                        \hline
                        1 & 0 & 1 & 1 & 0 \\
                        \hline
                        1 & 1 & 0 & 1 & 1 \\ [1ex]
                        \hline
                    \end{tabular}
                \end{center}
            \item \( (p \to q) \biconditional (\neg q \to \neg p) \) 
                \begin{center}
                    \begin{tabular}{||c c c c c c c||} 
                        \hline
                        \( p \)  & \( q \)  & \( \neg p \)  & \( \neg q\) & \( p \to q  \) & \( \neg q \to \neg p \) & \( (p \to q ) \biconditional (\neg q \to \neg p )  \)   \\ [0.5ex]
                        \hline\hline
                        0 & 0 & 1 & 1 & 1 & 1 & 1\\ 
                        \hline
                        0 & 1 & 1 & 0 & 1 & 1 & 1\\
                        \hline
                        1 & 0 & 0 & 1 & 0 & 0 & 1\\
                        \hline
                        1 & 1 & 0 & 0 & 1 & 1 & 1\\ [1ex]
                        \hline
                    \end{tabular}
                \end{center}
            \item \( (p \oplus q) \to (p \land q)  \) 
                \begin{center}
                    \begin{tabular}{||c c c c c ||} 
                        \hline
                        \( p \)  & \( q \) & \( p \oplus q  \) & \( p \land  q \) & \( (p \oplus q) \to (p \land q)  \)   \\ [0.5ex]
                        \hline\hline
                        0 & 0 & 0 & 0 & 1 \\ 
                        \hline
                        0 & 1 & 1 & 0 & 0 \\
                        \hline
                        1 & 0 & 1 & 0 & 0 \\
                        \hline
                        1 & 1 & 0 & 1 & 1 \\ [1ex]
                        \hline
                    \end{tabular}
                \end{center}
        \end{enumerate}
    \item 
        \begin{enumerate}
            \item \( 11000 \land (01011 \lor 11011 ) \) 

                \(01011 \lor 11011 = 11011\) 

                \( \Rightarrow 11011 \land 11000 = 11000\) 

            \item \( (01010 \oplus 11011) \oplus 01000\) 

                \( 01010 \oplus 11011 = 10001 \) 

                \( \Rightarrow 10001 \oplus 01000 = 11001\) 

        \end{enumerate}
    \item 
        Proposições atômicas:

        \begin{itemize}
            \item p := O sistema está no estado multiusuário.
            \item q := O sistema está operando normalmente.
            \item r := O kernel está funcionando.
            \item s := O sistema está no modo interrupção.
        \end{itemize}

        Proposições do sistema:
        \begin{itemize}
            \item \( p \biconditional q\) 
            \item \( p \to r \) 
            \item \( r \lor s \) 
            \item \( \neg p \to s \) 
            \item \( \neg s \) 
        \end{itemize}

        Fazendo a tabela verdade com os requisitos:
        \begin{center}
            \begin{tabular}{|c c c c c c c c c c|} 
                \hline
                \( p \)  & \( q \) & \( r \) & \( s \) & \( \neg p \) & \( \neg s \) & \( p \biconditional q \) & \( p \to r \) & \( r \lor s \) & \( \neg p \to s \)  \\ [0.5ex]
                \hline\hline
                0 & 0 & 0 & 0 & 1 & 1 & 1 & 1 & 0 & 0\\ 
                \hline
                0 & 0 & 0 & 1 & 1 & 0 & 1 & 1 & 1 & 1\\
                \hline
                0 & 0 & 1 & 0 & 1 & 1 & 1 & 1 & 1 & 0\\
                \hline
                0 & 0 & 1 & 1 & 1 & 0 & 1 & 1 & 1 & 1\\
                \hline
                0 & 1 & 0 & 0 & 1 & 1 & 0 & 1 & 0 & 0\\ 
                \hline
                0 & 1 & 0 & 1 & 1 & 0 & 0 & 1 & 1 & 1\\
                \hline
                0 & 1 & 1 & 0 & 1 & 1 & 0 & 1 & 1 & 0\\
                \hline
                0 & 1 & 1 & 1 & 1 & 0 & 0 & 1 & 1 & 1\\
                \hline
                1 & 0 & 0 & 0 & 0 & 1 & 0 & 0 & 0 & 1\\ 
                \hline
                1 & 0 & 0 & 1 & 0 & 0 & 0 & 0 & 1 & 1\\
                \hline
                1 & 0 & 1 & 0 & 0 & 1 & 0 & 1 & 1 & 1\\
                \hline
                1 & 0 & 1 & 1 & 0 & 0 & 0 & 1 & 1 & 1\\
                \hline
                1 & 1 & 0 & 0 & 0 & 1 & 1 & 0 & 0 & 1\\ 
                \hline
                1 & 1 & 0 & 1 & 0 & 0 & 1 & 0 & 1 & 1\\
                \hline
                1 & 1 & 1 & 0 & 0 & 1 & 1 & 1 & 1 & 1\\
                \hline
                1 & 1 & 1 & 1 & 0 & 0 & 1 & 1 & 1 & 1\\ [1ex]
                \hline
            \end{tabular}
        \end{center}

        Como a penúltima linha atende todos os requisitos, esse sistema \textbf{é consistente}.
\end{enumerate}

\end{document}
